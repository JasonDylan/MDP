\documentclass[a4paper,11pt]
{elsarticle}%EJOR要求The article should be A4 or letter size, in single column pages, with 11pt font.

 \usepackage{ctex}
%\usepackage{apalike}%加载APA样式包
\usepackage{hyperref}%超链接
\usepackage{multirow}%表格跨行
\usepackage{amsthm, amsmath}%数学公式
\usepackage{color}%颜色包
\usepackage{setspace}%间距宏包
%\usepackage{natbib}
\usepackage{mathrsfs}

\usepackage{cleveref}


%\usepackage{natbib}
\usepackage[backend=biber, style=apa]{biblatex}

%\bibliography{mybib.bib}
%\bibliographystyle{apalike}%使用APA样式
%\biboptions{authoryear}%这是在正文中的引用方式,使用作者名称及年份



%\usepackage{apacite}

%\newcommand*{\fullref}[1]{\namecref{#1} \nameref*{#1}}


\usepackage{amssymb,amsthm}
\usepackage{rotate,graphicx,epsfig}
\usepackage{algorithm}
\usepackage{algpseudocode}
\usepackage{caption}
\usepackage{multicol}
\usepackage{multirow}
\usepackage{arydshln}
\usepackage{longtable}
\def\bx{\expandafter\BX\expandafter{\ifnum0=`}\fi}
\newcommand{\nb}{\nonumber}
\newtheorem{theorem}{Theorem}[section]
\newtheorem{lemma}{Lemma}[section]
\newtheorem{corollary}{Corollary}[section]
\newtheorem{remark}{Remark}[section]
\newtheorem{definition}{Definition}[section]
\newtheorem{proposition}{Proposition}[section]
\newtheorem{example}{Example}[section]
\newtheorem{theo}{Theorem}[section]
\newtheorem{pr}{Proposition}[section]
\newtheorem{lem}{Lemma}[section]
\newtheorem{co}{Corollary}[section]
\newtheorem{re}{Remark}[section]
\newtheorem{de}{Definition}[section]
\newtheorem{exa}{Example}[section]
\newtheorem{mypro}{Proposition}
\newtheorem{mypre}{Theorem}
%\DeclareUnicodeCharacter{0301}{\'{e}}

%\biboptions{authoryear}
 
\onehalfspacing%EJOR要求 1.5 line spacing
 
%\journal{EJOR}

\bibliography{mybib.bib}
%\bibliographystyle{apalike}%使用APA样式

 
\begin{document}%文件开头
\section{代码对应函数}


\textbf{函数1}随机生成状态$S$。函数$S=f_1(I,L,W,M,x,\mathscr{H},\lambda)$。

参数$I$,$L$,$W$,$M$,$x$都是常数;向量$\mathscr{H}=(h_1,h_2,\ldots,h_M)$,其中$h_m\in\{1,\ldots,I\}$;矩阵$\lambda=(\lambda_{il})$是$I\times L$维矩阵。

函数用于随机生成状态$S$,$S=((n_{il}),(i_1,w_1),\ldots,(i_M,w_M))$,其中$(n_{il})$是$I\times L$维矩阵,且矩阵中的任意元素取值为$0,1,\ldots,x$,服从Possion分布,即对于$(n_{il})$中的第$i$行第$l$列的元素
$n_{il}$取值为$j=0,\ldots,x$的概率为
$$
P_j=\frac{\lambda_{il}^j}{j!}\cdot e^{-\lambda}
$$
这样便生成了$(n_{il})$。

对于$m=1,\cdots,M$,
$(i_m,w_m)$中$i_m\in\{1,2,\ldots,I\}$,均匀分布,即取到任何一个值的概率为$1/I$;$w_m\in\{0,1,2,\ldots,W\}$,取值也是均匀分布,$1/(W+1)$。特别注意:当$w_m=W$,则$i_m=h_m$是固定的。


这样给定各参数,就可以随机生成一个状态$S$。


~

~

\textbf{函数2}生成$S$到$\bar{S}$的函数。函数$\bar{S}=f_2(Z,S,X)$

参数:给定状态$S=((n_{il}),(i_1,w_1),\ldots,(i_M,w_M))$,和常数$Z$和$X$。

函数用于生成压缩后的状态$\bar{S}=((N)_{zl},(g_1,\ldots,g_Z),w)$,其中
$Z$将$L$切分为$Z$份,每一份有$L/Z$元素,如$L=15$,$Z=4$,则$\mathscr{Z}_1=\{1,2,3,4\}$,$\mathscr{Z}_2=\{5,6,7,8\}$,$\mathscr{Z}_3=\{9,10,11,12\}$,$\mathscr{Z}_4=\{13,14,15\}$四份,尽量平均分配,有余数则余数作为一类。

$(N_{zl})$是$Z\times L$维的矩阵,如$z=1$中第$l$列的元素为$N_{1l}=\max\{X,n_{1l}+n_{2l}+\ldots+n_{L/Z,l}\}$,表示在$\mathscr{Z}_1$这一份中的元素之和,以此类推。

令$e_z=\sum_{m=1}^M \mathbb{I}(i_m\in\mathscr{Z}_z)$ 是在聚类 $z$中的数量。 并且令 $\bar{M}=\sum_{m=1}^M \mathbb{I}(w_m\not=0)$ ,这里$\mathbb{I}(\cdot)$是指示函数,等于括号中的就为1,否则为0。

令 $g_z$ 满足:
\begin{align}\label{eq-g}
g_z=\left\{
\begin{array}{ll}
0,~~~&e_z=0;      \\1,~~~&e_z<\frac{\bar{M}}{Z};
\\
2,~~~&e_z\geq\frac{\bar{M}}{Z}.
\end{array}\right.
\end{align} 

最后,令$W\gets w_1$。

~

~


\textbf{函数3}生成$S$到$A$的函数: $A=f_3(S,\mathscr{L},\mathscr{H},r_1,(c_1(ij)),c_2)$

参数为$S=((n_{il}),(i_1,w_1),\ldots,(i_M,w_M))$和$\mathscr{L}=(l_1,\ldots,l_M)$,其中$l_m\in\{1,\ldots,L\}$;$\mathscr{H}=(h_1,h_2,\ldots,h_M)$;$r_1=(r_1(0),r_1(1),\ldots,r_1(L))$;$(c_1(i,j))$是$I\times I$的矩阵,$c_2$是常数。

状态生成决策$A=[(i^1,l^1),\ldots,(i^M,l^M)]$。其中$i^m\in\{1,2,\ldots,I\}$,$l^m\in\{1,2,\ldots,L\}$。

$A$中$(i^m,l^m)$需要满足如下约束条件:
\begin{alignat}{2}
&(i^m,l^m)=(h_m,0),~~~~~~ &if~ w_m=0, \label{s0-1}
\\
&(i^m,l^m)\in\{(i,l)|n_{il}>0,l\geq l_m\}\cup\{(h_m,0)\},~~~~~~~~&if~ w_m>0,   \label{s0-2}\\
&\phi_A(i,l):=\sum_{m}\mathbb{I}(i^m=i,l^m=l)\leq n_{il},~~~&\forall i,l.   \label{s0-3}
\end{alignat}

目标函数是:
\begin{align}\label{revenue}
R(S,A)=\max\sum\limits_{m}(r(l^m)-c_1(i_m,i^m))-c_2\sum\limits_{i,l}[n_{il}-\phi_A(i,l)],
\end{align}

这是一个线性规划问题,需要找到最优的决策$A$。


~


~

\textbf{函数4}生成$S$到$\mathcal{L}$的函数: $\mathcal{L}=f_4(S,\mathscr{L})$

参数$S,\mathscr{L}$以上都有介绍。

输出分类之后的$l$,如$\mathcal{L}=[(l(1),\ldots,l(k_1)),(l(k_1+1),\ldots,l(k_2)),\ldots,(k_n,\ldots,L)]$的形式类似,其中$k_1,\ldots,k_n$是根据$S$的变化而变化的,具体对应如下:

令$N_1(l=j)=\sum\limits_{m=1}^M\mathbb{I}(l_m=j,w_m\ne 0)$,其中$j=1,\ldots,L$,这里$N(l=j)$表示$l=j$的$m$的数量。
令$N_2(l=j)=\sum\limits_{k=1}^I(n_{il})_{kj}$,表示矩阵中第$j$列的加和,表示$l=j$的$n$的数量。

这样,我们假设$L=7$,则能够列出$N_1(l=1),\ldots,N_1(l=7)$和$N_2(l=1),\ldots,N_2(l=7)$如下表,tag用于比较每一行的大小,则有$6>5$,$6+2>2+1$,$6+2+3\leq 5+1+6$;遇到小于等于号则终止,将$l=1,2,3$归为一类;接着,$2>1$,$2+4\leq 1+17$,所以$l=4,5$归为一类;最后一类不论最后一个tag是$>$或$\leq$,都归为一类。所以在这个例子中的输出结果为$\mathcal{L}=[(1,2,3),(4,5),(6,7)]$。注意,若tag全为$>$,则只有一类$[(1,\ldots,L)]$。

\begin{table}[H]
  \centering
  \caption{example}\label{tab-exa}
  \begin{tabular}{c@{\hspace{1.2cm}}ccc}
\hline 
\text{$l$}  & 
\text{$N_1$} &  \text{tag}  &   \text{$N_2$} \\ \hline
1   &6 & $>$ & 5  \\
2 &2 & $>$ & 1 \\
3 &3 & $\leq$ & 6 \\
    \hdashline 4 &2 & $>$ & 1\\
5 &4 & $\leq$ & 17\\
   \hdashline 
6 &5 & $>$ & 4\\
7 &7 & $>$ & 3 
\\
\hline
\end{tabular}
\end{table}


\textbf{函数5} 生成$S$到$A=[A^1,\ldots,A^n]$的函数(这里的$A$和函数3中的$A$不一样):$A=f_5(S,\mathcal{L},N_1,N_2)$.

参数:这里利用了函数4得到的结果$N_1$和$N_2$,以及$\mathcal{L}$。

函数3是对所有的$m$做分配,而这里函数5是对在一个$\mathcal{L}$中的每个元组(类)做分配,例如在表1中的$l=1,2,3$,$l=2,4$以及$l=6,7$独立进行分配,分配的方式和函数3一致。

每个类的目标函数为:
\begin{align}
\label{R}
R^k(S,A^k)=&\max\sum\limits_{m\in\mathcal{M}^k(S)}\big[r(l^m)-c_1(i_m,i^m)\big]\nb\\
&-c_2\sum\limits_{i\in\mathcal{I},l\in\mathcal{L}^k}\big[n_{il}-\sum_{m\in\mathcal{M}^k(S)}\mathbb{I}(i^m=i,l^m=l)\big].
\end{align}


当每个类中的最后一个tag是$\leq$时,则有数学规划满足:
\begin{alignat}{2}
&(i^m,l^m)=(h_m,0),~~~~~~ &if~ w_m=0, \label{s1-1}
\\
%\label{s1-2}
&(i^m,l^m)\in\{(i,l)|n_{il}>0,l\geq l_m\},~~~~~~~~&if~ w_m>0,   \label{s1-2}\\
%\label{s0-3}
&\sum_{m\in\mathcal {M}^k(S)}\mathbb{I}(i^m=i,l^m=l)\leq n_{il},~~~~~~&\forall i\in\mathcal{I},l\in\mathcal{L}^k,   \label{s1-3}
\end{alignat}

当最后一个tag是$>$时,则有数学归纳满足:
\begin{alignat}{2}
%
&(i^m,l^m)=(h_m,0),~~~~~~ &if~ w_m=0, \label{s2-1}
\\
%\label{s1-2}
&(i^m,l^m)\in\{(i,l)|n_{il}>0,l\geq l_m\}\cup \{(h_m,0)\},~~~~~~~~&if~ w_m>0,   \label{s2-2}\\
%\label{s0-3}
&\sum_{m\in\mathcal {M}^k}\mathbb{I}(i^m=i,l^m=l)= n_{il},~~~&\forall i\in\mathcal{I},l\in\mathcal{L}^K.   \label{s2-3}
\end{alignat}
以上有三个变量需要解释:$A^k$,
$\mathcal{L}^K$和$\mathcal{M}^k(S)$。
其中$A^k=(i^m,l^m)_{m\in\mathcal{M}^k(S)}$,$\mathcal{M}^k(S)=\{m=1,
\ldots,M ~
|~ l_m
\in\mathcal{L}^k,w_m\ne 0\}$是在$
\mathcal{L}$中类$k$对应的$m$。
$
\mathcal{L}^k$是类$k$对应的$l$,如表1中$\mathcal{L}^1=(1,2,3)$;$\mathcal{L}^2=(4,5)$,$\mathcal{L}^3=(6,7)$。

最终对于任意的类$k$,得到$A^k=(i^m,l^m)_{m\in\mathcal{M}^k(S)}$。$A=[A^1,\ldots,A^n]$,实际上,这个也可以写成和函数3的输出一样的形式:$A=[(i^1,l^1),\ldots,(i^M,l^M)]$。

~

\textbf{函数6} 
函数5是输出最优决策,而另一个输出是决策空间$\mathcal{A}$。记为$\mathcal{A}=f_6(S,\mathcal{L},N_1,N_2)$。

$\mathcal{A}=[\mathcal{A}^1,\ldots,\mathcal{A}^L]$,其中
$\mathcal{A}^m=[(i_1^m,l_1^m),\ldots,(i_n^m,l_n^m)]$,表示所有$m=1$可能的决策,这个决策满足约束条件\eqref{s1-1}-\eqref{s2-3}。

换句话说函数$f_5$就是在$f_6$的基础上确定了最优的决策。

~

\textbf{函数7}
生成$\Xi=f_7(T,x)$

$T$和$x$是常数,$\Xi=((\xi_{il})_1,\ldots,(\xi_{il})_T)$。任意的$(\xi_{il})_t$是一个$I\times L$的矩阵,矩阵中的任意元素取值为$0,1,\ldots,x$,服从Possion分布,即对于$(\xi_{il})_t$中的第$i$行第$l$列的元素
$\xi_{il}$取值为$j=0,\ldots,x$的概率为
$$
P_j=\frac{\lambda_{il}^j}{j!}\cdot e^{-\lambda}
$$
这样便生成了$(n_{il})$。

~

\textbf{函数8(带起始探索的强化学习算法)}

生成最优值函数和最优策略:$[S_1,V^1,\pi^1]=f_8(f_1,V^0,\Xi,\pi^0)$。

参数介绍:$S_1$是由函数$f_1$随机生成的;$V^0=[V_1^0(S_1),V_2^0
,\ldots,V_T^0]$,除了$V_1^0(S_1)$是一个常数值,其他的均为向量:$V_t^0=(V_t^0(\bar{S}))_{\bar{S}\in\bar{\mathcal{S}}}$,其中,$\bar{\mathcal{S}}$是函数2中$\bar{S}$的状态空间,即$\bar{S}$所有可能的情况,这里设$V^0$中所有的元素都为0。令第0次迭代的策略
$\pi^0=[A^0_1,\ldots,A^0_T]$,
对于$t=1,2,\ldots,T$,满足
\begin{align}\label{A}
A_t^0=\operatorname*{arg\,max}\limits_{A_t\in f_6(S_t)}[{R}({S}_t,{A}_t)]
\end{align}
这里的$R(S_t,A_t)$是\eqref{revenue}中max内的部分(简化记为$R_t$)。
而在$t+1$阶段,
$S_{t+1}=S_t+(\xi_{il})_t$指的是$S_t$中的$(n_{il})$加上由函数7产生的$(\xi_{il})_t$部分。

步骤一:通过$f_1$随机生成一个初始状态$S_1$。

步骤二:计算在此状态下的最优决策:
$$
A_1=\arg\max\limits_{Y_1\in f_6}R(S_1,Y_1)+V^0_1(T(S_1,Y_1))
$$
这里,$R(S_1,Y_1)=R_1+R_2+\ldots-C_h$是在函数5中的每个等级类的目标函数相加,$f_6$是每个等级类中服务员的决策空间;初始令所有的$V^0=0$;$T(S_1,Y_1)=S_2$是转移状态,即$T(S_1,Y_1)=(S_1[0]+f_7-Y_1, i^1,w-1)$,城市中的任务有新到达的$f_7$,但是减掉已经完成的$Y_1$,员工的位置变为$i^1$,距离放假日减少1,但是如果$w=0$,则变成$w=W$。

步骤三:$S_2=T(S_1,Y_1)$应该通过$f_2$压缩,然后放入$V_1^0(f_2(T(S_1,Y_1)))$中,所以上式应该重新写为
$$
A_1=\arg\max\limits_{Y_1\in f_6}R(S_1,Y_1)+V_1^0(f_2(T(S_1,Y_1)))
$$

步骤四:计算这一步的收益$R_1(S_1,A_1)$,然后令$S_1\gets S_2$,回到步骤二。重复这个过程,一直算到最后一个阶段$T$,这样形成了整个链条:$S_1,A_1,R_1,S_2,A_2,R_2,\ldots,S_T$。

这个时候可以赋值$V^1$了,$$V^1_t(f_2(T(S_t,Y_t)))\gets R_{t+1}+\ldots+R_{T-1}$$,
特别的,$V^1_0(f_2(T(S_1,Y_1)))\gets R_{1}+\ldots+R_{T-1}$。

以上第一次迭代完成。

~

步骤五:回到步骤一,一直到步骤四进行第二次迭代,但是初始的值$V^0=0$这里不再是,而是要改为不全为0的,更新过的$V^1$。

同样形成了整个链条:$S_1,A_1,R_1,S_2,\ldots,S_T$,但是需要注意的是,如果本次迭代在某个$t$时刻的$f_2(S_t)$和第一次迭代是一样的,则在更新$V_{t-1}^2(f_2(S_t))=R_t+\ldots+R_{T-1}$需要改为$V_{t-1}^2(f_2(S_t))=\frac{1}{n}(R_t+\ldots+R_{T-1})+(1-\frac{1}{n})V_{t-1}^1(f_2(S_t))$,$n$表示$S_t$在迭代中重复的次数,这里重复两次所以是$n=2$,如果之后的迭代中又出现,就$n=3$,以此类推。

步骤六:在以上迭代了$J$次后,我们得到了$V^J$,这个时候对于任意的状态$S$,我们都能得到一个决策:
$$
A=\arg\max\limits_{Y\in f_6}R(S,Y)+V^J(f_2(T(S,Y)))
$$

以下介绍如何得到$V^1$和$\pi^1$。令$G_{t}=G_{t+1}+R_t$,从$T,\ldots,1$逆向进行计算,其中$G_{t+1}=0$,这样可以计算出所有的$G_t,t=1,\ldots,T$。
更新$V^1$如下:
\begin{align}\label{eq-updatev}
V_t^1(\bar{S}_t)=V_t^{0}(\bar{S}_t)+\frac{1}{\mathbb{N}(\bar{S}_t)}[G_t-V_t^{0}(\bar{S}_t)], 
\end{align}
这里$V_t^0(\bar{S}_t)$表示$V_t^0$这个向量中的第$\bar{S}_t$这个状态的值需要进行更新,更新为$V^1_t(\bar{S}_t)$,其余向量中的元素不变;$\frac{1}{\mathbb{N}(\bar{S}_t)}$表示$\bar{S}_t$这个状态在
$t$阶段之前迭代出现的次数,比如这是第一次迭代,肯定都是第一次出现,因此$\frac{1}{\mathbb{N}(\bar{S}_t)}=1$;随着出现次数的增加,$\frac{1}{\mathbb{N}(\bar{S}_t)}$的值越来越小。

更新$\pi^1=(A^1_1,\ldots,A^1_T)$如下,对于任意的阶段$t$,有:
\begin{align}\label{A}
A_t^1=\operatorname*{arg\,max}\limits_{A_t\in f_6(S_t)}[{R}({S}_t,{A}_t)+V^1_{t+1}(\overline{S_t+(\xi_{il})_t)}]
\end{align}
这里$\overline{S_t+(\xi_{il})_t}$表示状态$S_t$加上$(\xi_{il})_t)$之后形成下阶段状态后再进行状态压缩(函数2)。从而我们获得更新后的状态。

~

\textbf{函数9 }
按照函数8进行的规律学习
$[S_1,V^{j},\pi^j]=f_8(f_1,V^{j-1},\Xi,\pi^{j-1})$,只是将$V^0$变为$V^{j-1}$,$V^1$变为$V^{j-1}$。
$j=1,\ldots,J$。特别注意:在每次迭代中,第一个状态$S_1$都要重新根据$f_1$获取。

最终在第$J$次迭代后,我们获得$V^J=[V_1^J(S_1),V_2^J
,\ldots,V_T^J]$。

~

\textbf{函数10 }
$[V_t,A_t]=f_{10}(S_t)$

给定任意的状态$S_t$,能够得到该状态下的最优值$V_t$(一个常数)和最优策略$A_t$(一个分配策略),满足以下公式:
\begin{align}\label{A2}
A_t&=\operatorname*{arg\,max}\limits_{A_t\in f_6(S_t)}[{R}({S}_t,{A}_t)+V^J_{t+1}(\overline{S_t+(\xi_{il})_t)}]\nb\\
V_t&=\max\limits_{A_t\in f_6(S_t)}[{R}({S}_t,{A}_t)+V^J_{t+1}(\overline{S_t+(\xi_{il})_t)}]
\end{align}


~

基准策略:

\textbf{函数11(随机分配):}
$v_1=f_{11}(\Xi,S_t)$

表示给定$\Xi=((\xi_{il})_t,\ldots,(\xi_{il})_T)$和状态$S_t$就能够得到在状态$S_t$下的值$v_1$。

生成的轨迹如下:$S_t,A_t,R_t,(\xi_{il})_t,S_{t+1},\ldots,S_{T},A_{T},R_t$。
在这个轨迹中,$A_t$的确定是随机的,只需要$A_t=(i^m,l^m)$满足\eqref{s0-1}-\eqref{s0-3}中,随机选择一个决策即可。


\textbf{函数12(就近分配):}
$v_2=f_{12}(\Xi,S_t)$

生成的轨迹如下:$S_t,A_t,R_t,(\xi_{il})_t,S_{t+1},\ldots,S_{T},A_{T},R_t$。
在这个轨迹中,$A_t$的确定是就近的,需要$A_t=(i^m,l^m)$满足\eqref{s0-1}-\eqref{s0-3}的约束,同时有目标函数:
\begin{align}
    R^2=\min\sum\limits_{m=1}^M c_1(i_m,i^m)
\end{align}


\textbf{函数13(单阶段最优分配):}
$v_2=f_{12}(\Xi,S_t)$

生成的轨迹如下:$S_t,A_t,R_t,(\xi_{il})_t,S_{t+1},\ldots,S_{T},A_{T},R_t$。
在这个轨迹中,$A_t$的确定是就近的,需要$A_t=(i^m,l^m)$满足\eqref{s0-1}-\eqref{s0-3}的约束,同时有目标函数\eqref{revenue}。这个是保证每个阶段内是最优的决策。



\end{document}